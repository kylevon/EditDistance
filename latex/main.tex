%Configs
    \documentclass{article}
    \usepackage[utf8]{inputenc}
    %\usepackage[spanish]{babel}
    \usepackage{graphicx}
    \usepackage{hyperref}
    \graphicspath{ {./img/} }
    \usepackage{ amssymb }
    \usepackage{amsthm}
    \usepackage{amsmath}
    \usepackage{mathtools}
    \setlength{\parindent}{0pt}
    \newcommand{\abs}[1]{\left|#1\right|}
    \newcommand{\set}[1]{\left\{#1\right\}}
    \newcommand{\paren}[1]{\left(#1\right)}
    \newcommand{\st}{\ \vline \ }
    \usepackage{ stmaryrd }
    \newcommand{\dbrack}[1]{\left\llbracket#1\right\rrbracket}
    \newcommand{\exercise}[1]{\hrulefill \ \textsc{Edit distance}\ \hrulefill \newline}
    \newcommand{\trans}[1]{\overset{#1}{\longrightarrow}}
    \newcommand{\defeq}{\vcentcolon=}
    \newcommand{\ra}{\rightarrow}
    \newcommand{\lra}{\longrightarrow}
    \newtheorem{lemma}{Lemma}
    \newtheorem{theorem}{Theorem}
    \newtheorem{definition}{Definition}

    %\addtolength{\oddsidemargin}{0.7in}
    \addtolength{\topmargin}{-.875in}
	\addtolength{\textheight}{1.75in}
    \addtolength{\oddsidemargin}{-.875in}
	\addtolength{\evensidemargin}{-.875in}
	\addtolength{\textwidth}{1.75in}
    \marginparwidth=150pt


    \marginparwidth=150pt

\begin{document}

\begin{center}
    \Huge
    \textbf{Edit Distance}
    \vspace{10pt}
    \hrule
    \vspace{10pt}
    \huge
    \normalsize
\end{center}

\section{The Edit Distance Problem}

\subsection*{Definition of a ref-word}

Given a finite set of variables $V \subseteq \mathrm{SVars}$ we
define the alphabet of ref words as: $\Gamma_V \defeq \set{x\vdash,
\dashv x}$. And given an alphabet $\Sigma$ such that $\Sigma \cap
\Gamma_V = \varnothing$ we can define the set of ref words over
$\Sigma$ and $V$ as: $\mathbf{r} \in (\Sigma \cup \Gamma_V)^*$.
Next, a ref-word is valid if and only if, every occurance of a
variable in the ref-word is opened exactly once and closed afterwards,
exactly once.

\subsection*{Functions on ref-words}

We can define the projection of a ref word over a set $S$, $r\uparrow
S$, recursively as:\footnote{In the paper (Doleschal, 2021) this
operation is defined for $\Sigma$ as doc($\sigma$)}

\begin{enumerate}
    \item $r \in S \rightarrow r\uparrow S = r$
    \item $r \not\in S \rightarrow r \uparrow  S = \epsilon$
    \item $ (r_1 \cdot r_2) \uparrow  S = (r_1 \uparrow  S) \cdot (r_2 \uparrow  S)$
\end{enumerate}

Vars(r) is the set of variables $x \in V$ that occurs in the ref-word:

        \begin{equation}
	    \mathrm{Vars}(r) := \set{x \in V \st \exists r_x^{pre},
	    r_x, r_x^{post} \in (\Sigma \cup \Gamma_V)^* \mathrm{\
	    such\ that\ }r = r_x^{pre} \cdot x \vdash \cdot r_x
	    \cdot \dashv x \cdot r_x^{post})}
        \end{equation}

$\mathrm{tup}(r)$ are the positions each ref-word is referencing,
and is defined as:

\begin{equation}
    \mathrm{tup}(r) := \set{x \mapsto [i_x, j_x\rangle \st x \in
    Vars(R), i_x = \abs{r_x^{pre} \uparrow \Sigma}, j_x = i_x +
    \abs{r_x \uparrow \Sigma}}
\end{equation}

\textbf{Postulate:} 
\begin{equation}
    \mathrm{valid}(r) \rightarrow \abs{\mathrm{tup}(r)} = \abs{\mathrm{Vars}(r)}
\end{equation}
\subsection*{Definition of ref-word tuple}

\subsection*{Distance between two ref words}

We first define three posible ``edits'' on a word $w = w_1 \cdot\cdot\cdot w_n$:

\begin{enumerate}
    \item $(w, w') \in \Delta_{sub} \iff |w| = |w'| \land \exists i (w_i \not= w'_i \land \forall j \not= i(w_j = w'_j) \land w_i \in \Sigma \land w'_i \in \Sigma)$
    \item $(w, w') \in \Delta_{insert} \iff |w| = |w'| + 1 \land \exists i (\forall j < i(w_j = w'_j) \land \forall j \geq i (w_j = w'_{j+1}) \land w'_i \in \Sigma)$
    \item $(w, w') \in \Delta_{delete} \iff (w', w) \in \Delta_{insert}$
\end{enumerate}

Next, given the allowed edits $\Delta \subseteq \set{\Delta_{sub}, \Delta_{insert}, \Delta_{delete}}$. If there is a sequence of edits such that we obtain a sequence of words $w^1 , ..., w^n$ such that $(w^i, w^{i+1}) \in \bigcup \Delta$, then the length of the smallest of such sequences is the edit-distance of $w^1$ and $w^n$. Formally: 

\begin{equation}
    \mathbf{d}_\Delta(w, w') = min\paren{\set{ 
        \abs{seq} \st seq \in \set{
            (w^1,...,w^n) \st w^1 = w \land w^n = w' \land (w^i, w^{i+1}) \in \bigcup \Delta
        }
    }}
\end{equation}

If $\Delta = \set{\Delta_{sub}, \Delta_{insert}, \Delta_{delete}}$ then $\mathbf{d}_{\Delta}$ is the ref-word Levenshtein distance. $\mathbf{d}_{\Delta_{sub}}$ is the Hamming distance.
\\


Note that if we define the distance as $\mathbf{d_\Sigma}(r_1, r_2) = \mathbf{d}(r_1 \uparrow
\Sigma, r_2 \uparrow \Sigma)$, then the positions of all variables don't affect the distance.

\subsection*{Ref-word distance languages}

Given a ref-word language $R \subseteq (\Sigma\cup\Gamma_V)^*$ and
a distance $k \in \mathbb{R}$, the k-distance ref-word language is
defined as:

\begin{equation}
    R \pm k = \set{r \in (\Sigma \cup \Gamma_V)^* \st  \mathrm{valid(r)},
    \exists r' (r'\in R \land \mathbf{d_\Delta}(r, r') \leq k }
\end{equation}

Given a document d, the spanner over a ref-word language R is:

\begin{equation}
    \dbrack{R}_d = \set{\mathrm{tup}(r) \st r\uparrow \Sigma = d,
    \exists r \in R}
\end{equation}

\newpage
\begin{theorem}\label{TheoremMetric}
    If the distance function $\mathbf{d}$ is a metric, then:
    \begin{equation}
        (R\pm n) \pm k = R\pm(n + k)
    \end{equation}
    \begin{proof}
	Given $n, k \in \mathbb{R}$, we want to prove that: $(R\pm
	n )\pm k = R \pm (n + k)$.
    By definition, we have that:
    \begin{equation*}
        (R \pm n) \pm k = \set{r \in (\Sigma \cup \Gamma_V)^* \st  \mathrm{valid(r)}, \exists r' (r'\in R \pm n \land \mathbf{d_\Sigma}(r, r') \leq k)}
    \end{equation*}
    First we prove that $(R \pm n) \pm k \subseteq R \pm (n + k)$.
    By contradiction let's assume there exists an element $r_1 \in
    (R \pm n) \pm k$ such that $r_1 \notin R\pm (n + k)$. By the
    previous definition, we have that $\exists r' \in R\pm n$ such
    that $\mathbf{d_\Sigma}(r_1, r') \leq k$. By definition of $R
    \pm (n)$ we have that for any $r \in R$, $\mathbf{d_\Sigma}(r',
    r) \leq n$. Next, by the definition of $R \pm (n + k)$, and our
    suposition we have that $\mathbf{d_\Sigma}(r_1, r) > n + k$
    which contradicts the triangle inequality. The proof that $R
    \pm (n + k) \subseteq (R \pm n) \pm k$ uses this same argument.

    \end{proof}
\end{theorem}

\subsection*{Variable-set automaton over ref words (VSet-automaton)\footnote{(Doleschal, 2021)}}

\begin{definition}
A VSet-automaton is a sextuple $A:=(\Sigma, V, Q, q_0, Q_F,\delta)$
\begin{itemize}
    \item $\Sigma$: Alphabet symbols
    \item $V$: Finite set of variables
    \item $Q$: Finite set of states
    \item $q_0\in Q$: Initial state
    \item $Q_F\subseteq Q$: Set of final states
    \item $\delta: Q \times (\Sigma\cup\set{\epsilon}\cup\Gamma_V)\ra
          2^Q$: Transition function
        \begin{itemize}
            \item $\Gamma_V:=\set{x\vdash, \dashv x \st x \in V}$
            \item $2^Q$: power set of Q
        \end{itemize}
\end{itemize}
\end{definition}

%\subsubsection*{Functions over VSet-automaton}

%\begin{equation}
    %|A| = |Q| + |Q_F| + |\delta| + 1
%\end{equation}

%\begin{equation}
    %Vars(A) := V
%\end{equation}

\subsubsection*{Ref-word language}
The ref-word language of A is: $\mathcal{R}(A) = \mathcal{R}^0(A)
= \set{r \in \mathcal{L}(A)\subseteq(\Sigma\cup\Gamma_V)^*\st
r\mathrm{\ is\ accepted\ by\ the\ \epsilon-NFA\ A}}$. This is direct
from interpreting A as an $\epsilon-NFA$.

\subsubsection*{Run of a VSet-automaton over a ref-word}

Given a ref-word $r = \sigma_1\cdot\cdot\cdot\sigma_n$, the run
$\rho$ of $A$ is the sequence:

\begin{equation}
    \rho:=q_0\trans{\sigma_1}q_1 \cdot\cdot\cdot q_{n-1}
    \trans{\sigma_n}q_n
\end{equation}

Where $\forall i \in [0,n)\paren{q_{i+1} \in \delta(q_i,\sigma_{i+1})}$
and $q_n\in Q_F$

From previous publications we know that $r \in \mathcal{R}(A)$ if
and only if there is a run $\rho$ of A on r.

\newpage
\subsubsection*{Distance automaton}

Given a VSet-automaton $A$ we can define, under Levenshtein distance,
the automaton $A \pm 1:= (\Sigma, V, Q', q_0', Q_F', \delta')$ Where
\begin{itemize}
    \item $Q' = \set{q_1,...,q_{\abs{Q}}} \cup \set{q^{1}_1, ...,
    q^{1}_{\abs{Q}}} $ Where there exists two biyective functions:

        \begin{enumerate}
            \item $f: Q \ra \set{q_1,...,q_{\abs{Q}}}$ 
            \item $f' : Q \ra \set{q^{1}_1, ..., q^{1}_{\abs{Q}}}$
        \end{enumerate}

	And two biyective functions $F$ and $F'$ that map $f$ and
	$f'$ respectively to sets.

    \item $q_0' = f(q_0)$

    \item $Q_F' = Q_F \cup \set{q^1_i \st q_j \in Q_F \land f'(q_j) = q_i^1}$
    \item The function $\delta'$ is defined by:

        \begin{equation*}
        \delta'(q_i, e) = 
            \begin{cases}

            F(\delta(f^{-1}(q_i), e)) & e \in \Gamma_V\\
                    F(\delta(f^{-1}(q_i), e)) \cup q_i^1\cup
                    \underset{a \in \Sigma}{\bigcup}
                    F'(\delta(f^{-1}(q_i), a)) & e.o.c

            \end{cases}
        \end{equation*}

        \begin{equation*}
            \delta'(q_i^1, e) = F'(\delta(f'^{-1}(q_i^1), e))
        \end{equation*}

\end{itemize}
\begin{definition}\label{defSequential}
    A ref-word language $R$ is sequential if every ref-word $r \in
    R$ is valid.
\end{definition}

\begin{lemma}\label{LemmaTypesOfRuns}
    All runs of $\mathcal{R}(A\pm 1)$ have one of the following structures:
    \begin{enumerate}
        \item \begin{equation*}
            \rho_{A\pm 1} = f(q_0) \trans{c_1}
            f(\phi_1) \cdot\cdot\cdot
            \trans{c_{a-1}} f(\phi_{a-1})
            \trans{c_{a}} f'(\phi_{a-1})  \cdot
            \cdot \cdot \trans{c_n} f'(\phi_{n-1})
              \end{equation*}
        \item \begin{equation*}
            \rho_{A\pm 1} = f(q_0) \trans{c_1}
            f(\phi_1) \cdot\cdot\cdot\trans{c_{a-1}}
            f(\phi_a) \trans{c_{a}} f'(\phi_{a+1})
            \trans{c_{a+1}} f'(\phi_{a + 2})  \cdot
            \cdot \cdot \trans{c_n} f'(\phi_n)
              \end{equation*}
        \item \begin{equation*}
            \rho_{A\pm 1} = f(q_0) \trans{c_1}
            f(\phi_1) \cdot\cdot\cdot\trans{c_{a-1}}
            f(\phi_a) \trans{\epsilon} {f'(\phi_{a + 1})} \trans{c_{a}} f'(\phi_{a+2})
              \cdot
            \cdot \cdot \trans{c_n} f'(\phi_{n + 1})
              \end{equation*}
        \item \begin{equation*}
                \rho_{A\pm1} = f(q_0) \trans{c_1} f(\phi_1)
                \trans{c_2} \cdot\cdot\cdot
                \trans{c_n}f(\phi_n)
              \end{equation*}
    \end{enumerate}
    \begin{proof}

    This is direct from the form of the transitions under the definition of $A \pm 1$, and noticing that there is no transition from $f'(q)$ to $f(q)$.
    \end{proof}
\end{lemma}

\begin{theorem} \label{TheoremApm1Sequential}
    if $\mathcal{R}(A)$ if sequential, then $\mathcal{R}(A \pm 1)$.
    is sequential.
    
    \begin{proof}
	For purposes of contradiction let's assume that there exists
	a ref-word $r = c_1 \cdot \cdot\cdot c_n \in A \pm 1$ that
	is not valid. Then there must exist a run $\rho_{A\pm 1}$
    on r, and furthermore, by lemma \ref{LemmaTypesOfRuns}, $\rho_{A\pm1}$
    it can have only 4 structures. We will prove by enumeration that all
    these structures lead to a contradicton.
    \begin{enumerate}
        \item \textbf{Insertion}: By definition of the transitions
        of $A \pm 1$, we have that $c_1 \cdot\cdot\cdot c_{a-1}
        \cdot c_{a+1} \cdot\cdot\cdot c_{n} \in \mathcal{R}(A)$.
        This ref-word is valid by the premise of the theorem. Next,
        $c_a$ cannot be in $\Gamma_V$, and therefore we have a
        contradiction, because variables in all other characters
        open exactly once, and close exactly once only after they
        are opened by the definition of a valid ref-word.
        \item \textbf{Substitution, Elimination} Follow the same argument, in that the word changed is not in $\Gamma_V$.
        \item \textbf{No modification} Trivial.
        \end{enumerate}
    \end{proof}
\end{theorem}
\begin{theorem} \label{TheoremAutomatonPM1}
    Given a VSet-automaton with a sequential ref-word language
    $\mathcal{R}(A)$, using Levenshtein distance we obtain that
    $\mathcal{R}(A \pm 1) = \mathcal{R}(A) \pm 1$.

    \begin{proof}
	This is equivalent to proving that, given a ref-word $r$,
	$r \in \mathcal{R}(A \pm 1) \leftrightarrow r \in
	\mathcal{R}(A)\pm 1$. Therefore this is a two part proof.

	First let's assume that $r \in \mathcal{R}(A \pm 1)$. In
	that case we know that there must exist a run $\rho$ of $A
    \pm 1$ on r. r is valid due to the theorem \ref{TheoremApm1Sequential}.
        By Lemma \ref{LemmaTypesOfRuns} we have four cases for runs of $\mathcal{R}(a)$, We will now prove by enumeration on these cases:
    \begin{enumerate}
        \item \textbf{Insertion:} From the definition of $A \pm 1$, there must be a transition from $\phi_a$ to $\phi_{a+1}$ using the letter $a + 1$. Therefore, there is a run:
        \begin{equation*}
            q_0 \trans{c_1} \cdot\cdot\cdot\trans{c_{a-1}}\phi_{a-1} \trans{c_{a+1}} \phi_{a} \cdot\cdot\cdot \trans{c_n} \phi_{n - 1}
        \end{equation*}

	    And therefore, $c_1 \cdot\cdot\cdot c_a \cdot c_{a+2}
	    \cdot\cdot\cdot c_n \in \mathcal{R}(A)$. And because
	    the extended Levenshtein distance is of 1, and r is
	    valid, then $r \in \mathcal{R}(A) \pm 1$
            
    \item \textbf{Substitution, Elimination}: Follow the same argument, changing the sequence into a word and therefore, using the definitions of transitions arriving to a word in $\mathcal{R}(A)$.
        \item \textbf{No modification}: In this case, we can obtain a run over the same word
            in $\mathcal{R}(A)$, and therefore, by definition of $\mathcal{R}(A)
            \pm 1$ we obtain that $r \in \mathcal{R}(A)$
	    \end{enumerate}



        Next, let's assume that $r \in \mathcal{R}(A) \pm 1$. In that case, r is valid, and there exists $r' \in R$ such that using Levenshtein distance: $\mathbf{d_\Delta}(r, r') = \mathbf{d_\Delta}(r, r') \leq 1$. Since Levenshtein distance is discrete, there are two cases:
        \begin{enumerate}
            \item $\mathbf{d_\Sigma}(r, r') = 0$. Then it is clear
            that $r \in \mathcal{R}(A \pm 1)$ since a subset of $A
            \pm 1$ forms an isomorphism with $A$.

            \item $\mathbf{d_\Sigma}(r, r') = 1$. In this case,
            because of the structure of Levenshtein's distance,
            there are three possible cases:

            \begin{itemize}
                \item \textbf{Insertion}. In this case the word r looks like:
		    $c_1 \cdot\cdot\cdot c_i \cdot c_{inserted}
		    \cdot c_{i + 1} \cdot\cdot\cdot c_n$. Furthermore,
		    $c_{inserted} \in \Sigma$ by definition.
		    By definition we obtain that $c_1\cdot\cdot\cdot c_n \in
		    \mathcal{R}(A)$ and therefore, this word is accepted by the
		    automaton $A\pm1$.
                \item \textbf{Substitution / Elimination} Same arguments as Insertion.
            \end{itemize}
        \end{enumerate}
    \end{proof}
\end{theorem}

\begin{theorem}
    For any sequential automatons $A$ there exists an automaton B such that $\mathcal{R}(B) = \mathcal{R(A)} \pm k$ for all $k\in \mathbb{N}$ under Levenshtein distance.
    \begin{proof}
        Trivial proof using theorems \ref{TheoremMetric} and \ref{TheoremAutomatonPM1}, and induction over k.
    \end{proof}
\end{theorem}


\section{Alternative Definitions}

\subsection{Index Approximate String Matching}

Given a text $\mathcal{T}$ , a regular expression $\mathcal{E}$,
an edit distance between strings $d()$, and a distance threshold
$k$, find all the text positions that start an approximate occurrence
of $\mathcal{E}$ in $\mathcal{T}$, that is, compute the set:
\begin{equation*}
    \set{i, \exists j, \exists S \in L(\mathcal{E}), d(\mathcal{T}_{i...j} , S) \leq k}
\end{equation*}

This definition is used in (Navarro, 2001), (Kucherov, 2015)

\subsection{Substring Approximate String Matching}

Given a text $\mathcal{T}$ , a regular expression $\mathcal{E}$,
an edit distance between strings $d()$, and a distance threshold
$k$, find all the substrings that in T that are have less than k
distance from $\mathcal{E}$. That is:
\begin{equation*}
    \set{\mathcal{T}_{i...j}, \exists i, \exists j, \exists S \in L(\mathcal{E}), d(\mathcal{T}_{i...j} , S) \leq k}
\end{equation*}

This definition is used in: (Wu, 1992), (Kucherov, 2005), (Farach-Colton, 2007), (Karkkainen, 2007)

\subsection{All-match Approximate String Join}
Given two string lists $\mathcal{S}$ and $\mathcal{T}$, a similarity measure $sim(S, T)$ and a real number $\theta \in [0,1]$. Find a set of string pairs $\mathcal{R}$ consisting of all string pairs $(S, T) \in \mathcal{S} \times \mathcal{R}$ where $sim(S,T) \geq \theta$ holds.

Defined in the thesis (xu, 2021)

\subsection{Top-k Approximate String Join}

Given two string lists $\mathcal{S}$ and $\mathcal{T}$, a similarity measure $sim(S, T)$ and a positive integer k. Find a set of string pairs $\mathcal{R}$ of at most k items, such that every $(S,T) \in \mathcal{R}$ satisfies $sim(S,T) \geq sim(P, Q)$ where (P, Q) can be any string pair from $\mathcal{S} \times \mathcal{T} \backslash R$

Defined in the thesis (xu, 2021)

\subsection{K mismatches variation}
The pattern P' is an occurance of $P = p_0 \cdot \cdot\cdot p_{m-1}$ if $|P| = |P'|$ and P' has at most k mismatches with P. Find all occurances of $P$ in the document T.

This definition is used in (Fiori, 2021)

\subsection{Brainstorming - Vrgoc}
\begin{equation*}
    \set{(i, j, k') \st \exists S \in L(\mathcal{E}), d(\mathcal{T}_{i...j},S) = k' \leq k}
\end{equation*}
But instead returning the tup(r)'s of the whole match.

\section{Possible extensions}

Since there are so many possible results obtained through this method, it would be useful to only return optimal results. One possibility is to return the local minimums. That is, the tuples where the distance is minimized with respect to their neighbors.

\end{document}
