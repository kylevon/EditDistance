%Configs
    \documentclass{article}
    \usepackage[utf8]{inputenc}
    %\usepackage[spanish]{babel}
    \usepackage{graphicx}
    \usepackage{hyperref}
    \graphicspath{ {./img/} }
    \usepackage{ amssymb }
    \usepackage{amsthm}
    \usepackage{amsmath}
    \usepackage{mathtools}
    \setlength{\parindent}{0pt}
    \newcommand{\abs}[1]{\left|#1\right|}
    \newcommand{\set}[1]{\left\{#1\right\}}
    \newcommand{\paren}[1]{\left(#1\right)}
    \newcommand{\st}{\ \vline \ }
    \usepackage{ stmaryrd }
    \newcommand{\dbrack}[1]{\left\llbracket#1\right\rrbracket}
    \newcommand{\exercise}[1]{\hrulefill \ \textsc{Edit distance}\ \hrulefill \newline}
    \newcommand{\defeq}{\vcentcolon=}
    \newcommand{\ra}{\rightarrow}
    \newcommand{\lra}{\longrightarrow}
    \newtheorem{lemma}{Lemma}
    \newtheorem{theorem}{Theorem}
    \newtheorem{definition}{Definition}

    %\addtolength{\oddsidemargin}{0.7in}
    \addtolength{\topmargin}{-.875in}
	\addtolength{\textheight}{1.75in}
    \addtolength{\oddsidemargin}{-.875in}
	\addtolength{\evensidemargin}{-.875in}
	\addtolength{\textwidth}{1.75in}
    \marginparwidth=150pt


    \marginparwidth=150pt

\begin{document}

\begin{center}
    \Huge
    \textbf{Edit Distance}
    \vspace{10pt}
    \hrule
    \vspace{10pt}
    \huge
    \normalsize
\end{center}

\section{The Edit Distance Problem}

\subsection*{Definition of a ref-word}

Given a finite set of variables $V \subseteq \mathrm{SVars}$ we define the alphabet of ref words as: $\Gamma_V \defeq \set{x\vdash, \dashv x}$. And given an alphabet $\Sigma$ such that $\Sigma \cap \Gamma_V = \varnothing$ we can define the set of ref words over $\Sigma$ and $V$ as: $\mathbf{r} \in (\Sigma \cup \Gamma_V)^*$. Next, a ref-word is valid if and only if, every occurance of a variable in the ref-word is opened exactly once and closed afterwards, exactly once.

\subsection*{Functions on ref-words}

We can define the projection of a ref word over a set $S$, $r\uparrow S$, recursively as:\footnote{In the paper (Doleschal, 2021) this operation is defined for $\Sigma$ as doc($\sigma$)}

\begin{enumerate}
    \item $r \in S \rightarrow r\uparrow S = r$
    \item $r \not\in S \rightarrow r \uparrow  S = \epsilon$
    \item $ (r_1 \cdot r_2) \uparrow  S = (r_1 \uparrow  S) \cdot (r_2 \uparrow  S)$

\end{enumerate}

Vars(r) is the set of variables $x \in V$ that occurs in the ref-word:

        \begin{equation}
            \mathrm{Vars}(r) := \set{x \in V \st \exists r_x^{pre}, r_x, r_x^{post} \in (\Sigma \cup \Gamma_V)^* \mathrm{\ such\ that\ }r = r_x^{pre} \cdot x \vdash \cdot r_x \cdot \dashv x \cdot r_x^{post})}
        \end{equation}

$\mathrm{tup}(r)$ are the positions each ref-word is referencing, and is defined as:

\begin{equation}
    \mathrm{tup}(r) := \set{x \mapsto [i_x, j_x\rangle \st x \in Vars(R), i_x = \abs{r_x^{pre} \uparrow \Sigma}, j_x = i_x + \abs{r_x \uparrow \Sigma}}
\end{equation}

\textbf{Postulate:} 
\begin{equation}
    \mathrm{valid}(r) \rightarrow \abs{\mathrm{tup}(r)} = \abs{\mathrm{Vars}(r)}
\end{equation}
\subsection*{Definition of ref-word tuple}

\subsection*{Distance between two ref words}


Next, given two ref words $r_1, r_2 \in (\Sigma \cup \Gamma_V)$ and
a distance function 
$\mathbf{d} : \Sigma^* \times \Sigma^* \rightarrow \mathbb{R}$
the distance $\mathbf{d_\Gamma}$ between $r_1$ and $r_2$
is defined as: $\mathbf{d_\Gamma}(r_1, r_2) = \mathbf{d}(r_1 \uparrow
\Sigma, r_2 \uparrow \Sigma)$

\subsection*{Ref-word distance languages}

Given a ref-word language $R \subseteq (\Sigma\cup\Gamma_V)^*$ and a distance $k \in \mathbb{R}$, the k-distance ref-word language is defined as:

\begin{equation}
    R \pm k = \set{r \in (\Sigma \cup \Gamma_V)^* \st  \mathrm{valid(r)}, \exists r' (r'\in R \land \mathbf{d_\Gamma}(r, r') \leq k)}
\end{equation}

Given a document d, the spanner over a ref-word language R is:

\begin{equation}
    \dbrack{R}_d = \set{\mathrm{tup}(r) \st r\uparrow \Sigma = d, \exists r \in R}
\end{equation}

\newpage
\begin{theorem}
    If the distance function $\mathbf{d}$ is a metric, then:
    \begin{equation}
        (R\pm n) \pm k = R\pm(n + k)
    \end{equation}
    \begin{proof}
        Given $n, k \in \mathbb{R}$, we want to prove that: $(R\pm n )\pm k = R \pm (n + k)$.
        By definition, we have that:
        \begin{equation*}
            (R \pm n) \pm k = \set{r \in (\Sigma \cup \Gamma_V)^* \st  \mathrm{valid(r)}, \exists r' (r'\in R \pm n \land \mathbf{d_\Gamma}(r, r') \leq k)}
        \end{equation*}
        First we prove that $(R \pm n) \pm k \subseteq R \pm (n + k)$. By contradiction let's assume there exists an element $r_1 \in (R \pm n) \pm k$ such that $r_1 \notin R\pm (n + k)$. By the previous definition, we have that $\exists r' \in R\pm n$ such that $\mathbf{d_\Gamma}(r_1, r') \leq k$. By definition of $R \pm (n)$ we have that for any $r \in R$, $\mathbf{d_\Gamma}(r', r) \leq n$. Next, by the definition of $R \pm (n + k)$, and our suposition we have that $\mathbf{d_\Gamma}(r_1, r) > n + k$ which contradicts the triangle inequality. The proof that $R \pm (n + k) \subseteq (R \pm n) \pm k$ uses this same argument.

    \end{proof}
\end{theorem}


% Sea A vset automaton. 
% Exists A^k s.t. 
% Def lenguaje autómata
% [[B]_d = [[L(A)]]]^k_d
% Definir con vset automata. con el de ref words
% Definir L(A)
% Definir functional, sequential
% Demostrar [[B]]_d = [[L(A)]]^k_d.
%  Al definir B con los edit distances
% Demostrar que si A es functional
% B es functional también.

% Tamaño del determinación
% Ver las transiciones con cierra y abre x's

\subsection*{Variable-set automaton over ref words (VSet-automaton)\footnote{(Doleschal, 2021)}}


\begin{definition}
A VSet-automaton is a sextuple $A:=(\Sigma, V, Q, q_0, Q_F,\delta)$
\begin{itemize}
    \item $\Sigma$: Alphabet symbols
    \item $V$: Finite set of variables
    \item $Q$: Finite set of states
    \item $q_0\in Q$: Initial state
    \item $Q_F\subseteq Q$: Set of final states
    \item $\delta: Q \times (\Sigma\cup\set{\epsilon}\cup\Gamma_V)\ra 2^Q$: Transition function
        \begin{itemize}
            \item $\Gamma_V:=\set{x\vdash, \dashv x \st x \in V}$
            \item $2^Q$: power set of Q
        \end{itemize}
\end{itemize}
\end{definition}

\subsubsection*{Functions over VSet-automaton}

\begin{equation}
    |A| = |Q| + |Q_F| + |\delta| + 1
\end{equation}

\begin{equation}
    Vars(A) := V
\end{equation}

\subsubsection*{Ref-word language}
The ref-word language of A is: $\mathcal{R}(A) = \mathcal{R}^0(A) = \set{r \in \mathcal{L}(A)\subseteq(\Sigma\cup\Gamma_V)^*\st r\mathrm{\ is\ accepted\ by\ the\ \epsilon-NFA\ A}}$. This is direct from interpreting A as an $\epsilon-NFA$.

\subsubsection*{Run of a VSet-automaton over a ref-word}

Given a ref-word $r = \sigma_1\cdot\cdot\cdot\sigma_n$, the run $\rho$ of $A$ is the sequence:

\begin{equation}
    \rho:=q_0\overset{\sigma_1}{\lra}q_1 \cdot\cdot\cdot q_{n-1} \overset{\sigma_n}{\lra}q_n
\end{equation}

Where $\forall i \in [0,n)\paren{q_{i+1} \in \delta(q_i,\sigma_{i+1})}$ and $q_n\in Q_F$ 

From previous publications we know that $r \in \mathcal{R}(A)$ if and only if there is a run $\rho$ of A on r.

\newpage
\subsubsection*{Distance automaton}

Given a VSet-automaton $A$ we can define, under Levenshtein distance, the automaton $A \pm 1:= (\Sigma, V, Q', q_0', Q_F', \delta')$ Where
\begin{itemize}
    \item $Q' = \set{q_1,...,q_{\abs{Q}}} \cup \set{q^{1}_1, ..., q^{1}_{\abs{Q}}} $ Where there exists two biyective functions:

        \begin{enumerate}
            \item $f: Q \ra \set{q_1,...,q_{\abs{Q}}}$ 
            \item $f' : Q \ra \set{q^{1}_1, ..., q^{1}_{\abs{Q}}}$
        \end{enumerate}

        And two biyective functions $F$ and $F'$ that map $f$ and $f'$ respectively to sets.

    \item $q_0' = f(q_0)$

    \item $Q_F' = Q_F \cup \set{q^1_i \st q_j \in Q_F \land f'(q_j) = q_i^1}$
    \item The function $\delta'$ is defined by:

        \begin{equation*}
        \delta'(q_i, e) = \begin{cases}
            F(\delta(f^{-1}(q_i), e)) & e \in \Gamma_V\\
                            F(\delta(f^{-1}(q_i), e)) \cup q_i^1\cup \underset{a \in \Sigma}{\bigcup} F'(\delta(f^{-1}(q_i), a)) & e.o.c
                          \end{cases}
        \end{equation*}
        \begin{equation*}
            \delta'(q_i^1, e) = F(\delta(f^{-1}(q_i), e))
        \end{equation*}
\end{itemize}
\begin{theorem}
    Given a VSet-automaton with a sequential ref-word language $\mathcal{R}(A)$, using Levenshtein distance we obtain that $\mathcal{R}(A \pm 1) = \mathcal{R}(A) \pm 1$

    \begin{proof}
        This is equivalent to proving that, given a refword $r$, $r \in \mathcal{R}(A \pm 1) \leftrightarrow r \in \mathcal{R}(A)\pm 1$. Therefore this is a two part proof.

        First let's assume that $r \in \mathcal{R}(A \pm 1)$. In that case we know that there must exist a run $\rho$ of $A \pm 1$ on r. First, by definition of $\delta'$ there is no transition from states of the form $q_i^1$ to $q_i$. Therefore, there are two cases:
        \begin{enumerate}
            \item There are only states of the form $q_i$. In this case, the subset of nodes and transitions used have an isomorphism to $A$ using the function $f$, therefore, $r \in \mathcal{R}(A)$, and therefore $r \in \mathcal{R}(A) \pm 1$
            \item There is one transition from a state of the form $q_i$ to $q_i^1$. The respective run of $A \pm 1$ on $r = c_1 \cdot \cdot \cdot c_n$ could have one of three forms:

 \textbf{Insertion}
            \begin{equation}
                \rho_{A\pm 1} = f(q_0) \overset{c_1}{\lra} f(\phi_1) \cdot\cdot\cdot\overset{c_{i-1}}{\lra} f(\phi_i) \overset{c_{i}}{\lra} f'(\phi_i) \overset{c_{i+1}}{\lra} f'(\phi_{i + 1})  \cdot \cdot \cdot \overset{c_n}{\lra} f'(\phi_n)
            \end{equation}
        \end{enumerate}

    In this case, there exists a run:

        \begin{equation}
            \rho_{A\pm 1} = q_0 \overset{c_1}{\lra} \phi_1 \cdot\cdot\cdot \overset{c_{i-1}}{\lra}\phi_i  \overset{c_{i+1}}{\lra} \phi_{i + 1}  \cdot \cdot \cdot \overset{c_n}{\lra} \phi_n
        \end{equation}

        Such that every transition is in $A$, therefore the word $r'=c_1\cdot\cdot\cdot c_{i-1} \cdot c_{i+1} \cdot \cdot\cdot c_n$ is in $\mathcal{R}(A)$. And therefore by Levenshtein distance we obtain that $\mathbf{d_\Gamma}(r',r) = 1$. Andtherefore, $r \in \mathcal{R}(A) \pm 1$

\textbf{Substitution}

TODO

        \textbf{Elimination}

        TODO

        Next, let's assume that $r \in \mathcal{R}(A) \pm 1$. In that case, r is valid, and there exists $r' \in R$ such that using Levenshtein distance: $\mathbf{d_\Gamma}(r, r') \leq 1$. Since Levenshtein distance is discrete, there are two cases:
        \begin{enumerate}
            \item $\mathbf{d_\Gamma}(r, r') = 0$. Then it is clear that $r \in \mathcal{R}(A \pm 1)$ since a subset of $A \pm 1$ forms an isomorphism with $A$.
            \item $\mathbf{d_\Gamma}(r, r') = 1$. In this case, because of the structure of Levenshtein's distance, there are three possible cases:
                \begin{enumerate}
                    \item \textbf{Insertion} $ r = c_1 \cdot ... \cdot c_i \cdot c_{inserted} \cdot c_{i + 1} \cdot ... \cdot c_n$  And $r' = c_1 \cdot ... \cdot c_n$

                        In this case, because there exists a run $\rho_A$ on $r'$ of the form:
                        \begin{equation*}
                            \rho_A = q_0 \overset{c_1}{\lra} \phi_1 \cdot\cdot\cdot \phi_i \overset{c_i}{\lra} \phi_{i + 1}  \cdot \cdot \cdot \overset{c_n} \phi_n
                        \end{equation*}

                        Therefore there exists a run $\rho_{A\pm1}$ on $r$, and it is:

                        \begin{equation*}
                            \rho_{A\pm 1} = f(q_0) \overset{c_1}{\lra} f(\phi_1) \cdot\cdot\cdot f(\phi_i) \overset{c_{ins}}{\lra} f'(\phi_i) \overset{c_i}{\lra} f'(\phi_{i + 1})  \cdot \cdot \cdot \overset{c_n}{\lra} f'(\phi_n)
                        \end{equation*}
                    \item \textbf{Elimination } TODO, Same as above
                    \item \textbf{Substitution} TODO, Same as above
                \end{enumerate}
        \end{enumerate}
        % Interpreting $A\pm 1$ as an $\epsilon - NFA$, we can conclude that for every $r \in \mathcal{R}(A\pm 1)$ there is a run $\rho$ of $A$ on $r$.
    \end{proof}
\end{theorem}

\end{document}
